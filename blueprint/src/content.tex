% In this file you should put the actual content of the blueprint.
% It will be used both by the web and the print version.
% It should *not* include the \begin{document}
%
% If you want to split the blueprint content into several files then
% the current file can be a simple sequence of \input. Otherwise It
% can start with a \section or \chapter for instance

\section{Normality and Ultraweak Continuity}

In what follows, let $\mathcal{M}$ be a $W^{*}$--algebra. Let $\mathcal{P}(\mathcal{M})$
denote the projection lattice of $\mathcal{M}$.

\begin{lemma}
  For all self-adjoint $x\in \mathcal{M}$, $\|x\|1-x \ge 0$.
\end{lemma}

\begin{lemma}
  For all $p,q\in \mathcal{P}(\mathcal{M})$ such that $q\le p$, $p-q\in \mathcal{P}(\mathcal{M})$.
\end{lemma}



\section{Continuous functional calculus}

\begin{definition}[Continuous functional calculus]
  \label{def:continuous_functional_calculus}
  \lean{ContinuousFunctionalCalculus}
  \mathlibok
  A $*$-$R$-algebra is said to have a continuous functional calculus for elements satisfying a predicate $p$ if,
  for each $a$ satisfying $p$, there is a $*$-homomorphism $\phi_a : C(\spectrum[R]{a}, R) \to A$ sending the identity function to $a$, and which is a closed embedding.
  Moreover, $\spectrum[R]{a}$ is compact and nonempty, and $\phi_a$ satisfies the spectral mapping property (i.e., $\spectrum[R]{\phi_a(f)} = f(\spectrum[R]{a})$).
\end{definition}

\begin{definition}[Non-unital continuous functional calculus]
  \label{def:nonunital_continuous_functional_calculus}
  \lean{ContinuousFunctionalCalculus}
  \mathlibok
  A non-unital $*$-$R$-algebra is said to have a non-unital continuous functional calculus for elements satisfying a predicate $p$ if,
  for each $a$ satisfying $p$, there is a non-unital $*$-homomorphism $\phi_a : C(\quasispectrum[R]{a}, R)_0 \to A$ (here $C(\quasispectrum[R]{a}, R)_0$ is the collection of functions vanishing at zero on the quasispectrum) sending the identity function to $a$, and which is a closed embedding.
  Moreover, $\quasispectrum[R]{a}$ is compact (it's always nonempty because it contains $0$), and $\phi'_a$ satisfies the spectral mapping property (i.e., $\quasispectrum[R]{\phi'_a(f)} = f(\quasispectrum[R]{a})$).
\end{definition}

\begin{definition}
  \label{def:cfc}
  \lean{cfc}
  \mathlibok
  \uses{def:continuous_functional_calculus}
  Given $a \in A$ satisfying $p$ and $f : R → R$ continuous on $\spectrum[R]{a}$, we define $f(a) := \phi'_a(f)$ (and we give it a junk value of zero when either $a$ does not satisfy $p$ or $f$ is not continuous on the spectrum).
\end{definition}

\begin{definition}
  \label{def:cfcₙ}
  \lean{cfcₙ}
  \mathlibok
  \uses{def:nonunital_continuous_functional_calculus}
  Given $a \in A$ satisfying $p$ and $f : R → R$ continuous on $\spectrum[R]{a}$ and $f(0) = 0$, we define $f(a) := \phi'_a(f)$ (and we give it a junk value of zero when and of the conditions on $a$ and $f$ are not met).
\end{definition}

\begin{theorem}
  \label{thm:gelfand_cfc_iso}
  \lean{continuousFunctionalCalculus}
  \mathlibok
  For every normal element $a$ in a unital $C^*$-algebra $A$ there is a $*$-isomorphism between $C(\spectrum{a}, ℂ)$ and the $C^*$-subalgebra of $A$ generated by $a$.
\end{theorem}

